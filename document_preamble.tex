%======================================================================
% Include packages
%----------------------------------------------------------------------
\usepackage{algorithm}
\usepackage[noend]{algpseudocode}
\usepackage{amsmath}
\usepackage{amssymb}
\usepackage{amsthm}
\usepackage{bbm}
\usepackage{cancel}
\usepackage{csquotes}
\usepackage{etoolbox} %% <- for \pretocmd and \apptocmd
\usepackage[margin=1in]{geometry} %% \usepackage{fullpage}
\usepackage{graphicx}
%% \usepackage{IEEEtrantools} %% uncomment if you use this environment
\usepackage{lineno}
\usepackage{lipsum}
\usepackage{longtable}
\usepackage{makecell}
\usepackage{mathtools} %% for \bigtimes
\usepackage{multirow}
\usepackage{nicefrac}
\usepackage{setspace}
\usepackage{thmtools, thm-restate}
\usepackage{titling}
\usepackage[nottoc]{tocbibind} % for adding list of figures to the table of contents
\usepackage[normalem]{ulem}
\usepackage{xcolor}

%% for incorporating extra symbols
\usepackage{skull} %% remember, the symbol 'C' is much better

\RequirePackage[hyphens]{url}
\PassOptionsToPackage{hyphens}{url}
\usepackage{hyperref}
%% https://tex.stackexchange.com/questions/823/remove-ugly-borders-around-clickable-cross-references-and-hyperlinks
%% \usepackage[hidelinks]{hyperref}
\hypersetup{
  colorlinks,
  linkcolor={red!33!black},
  citecolor={blue!33!black},
  urlcolor={blue!33!black}
}

%% for using biblatex (https://tug.ctan.org/info/biblatex-cheatsheet/biblatex-cheatsheet.pdf)
%% Also, see Section 3 of https://mirror.csclub.uwaterloo.ca/CTAN/macros/latex/contrib/biblatex/doc/biblatex.pdf
%% \usepackage{babel,csquotes,xpatch}
%% \usepackage[backend=biber, style=authoryear-comp, giveninits=true]{biblatex}
%% \DeclareNameAlias{author}{last-first} %% https://tex.stackexchange.com/questions/290103/get-last-name-first-in-biblatex-using-numeric-style
\NewDocumentCommand{\bib}{}{\hyperlink}

%% for showing the label keys in the document margins
%% adapted from https://tex.stackexchange.com/questions/148579/tweak-showlabels-showkeys-wrap-the-label
\usepackage{seqsplit}
\usepackage[notref, notcite]{showkeys}
\usepackage{xstring}
\renewcommand*\showkeyslabelformat[1]{%
  \noexpandarg%
  \StrSubstitute{\(\{\)#1\(\}\)}{ }{\textvisiblespace}[\TEMP]%
  \framebox{\parbox[c]{\marginparwidth}{\raggedright\tiny\ttfamily\expandafter\seqsplit\expandafter{\TEMP}}}}

%======================================================================
% Math declarations
%----------------------------------------------------------------------
%% for help with these, look at the thmtools-manual.pdf
\declaretheoremstyle[bodyfont=\normalfont, shaded={rulecolor=black,
rulewidth=0.5pt, bgcolor=black!5!white}]{normalfont}
\declaretheorem[numberwithin=section, style=normalfont]{theorem, assumption}
\declaretheorem[numberlike=theorem, style=normalfont]{lemma, proposition, example, corollary, definition, remark, observation}

\NewDocumentCommand{\fclrbx}{mm}{\fcolorbox{#1!67!black}{#1!5!white}{#2}} %% https://tex.stackexchange.com/questions/446548/how-to-change-the-default-border-color-of-fbox

%% copied from /usr/local/texlive/2023/texmf-dist/tex/latex/amscls/amsthm.sty
%% changed the \itshape in the original command to
%% for explanation about the commands \makeatletter or \makeatother, see https://tex.stackexchange.com/questions/8351/what-do-makeatletter-and-makeatother-do
\makeatletter
\newenvironment{boldproof}[1][\proofname]{\par
  \pushQED{\qed}%
  \normalfont \topsep6\p@\@plus6\p@\relax
  \trivlist
  \item[\hskip\labelsep
        \bfseries % \itshape
    #1\@addpunct{.}]\ignorespaces
}{%
  \popQED\endtrivlist\@endpefalse
}
\makeatother

\DeclareMathOperator{\union}{\cup}
\DeclareMathOperator{\intersect}{\cap}
\DeclareMathOperator{\Union}{\bigcup}
\DeclareMathOperator{\Intersect}{\bigcap}

\NewDocumentCommand{\eps}{}{\varepsilon} %% \NewDocumentCommand{\vareps}{\varepsilon}

\DeclareMathOperator*{\linf}{\lim\inf}
\DeclareMathOperator*{\lsup}{\lim\sup}
\DeclareMathOperator*{\argsup}{\arg\sup}
\DeclareMathOperator*{\arginf}{\arg\inf}
\DeclareMathOperator*{\argmax}{\arg\max}
\DeclareMathOperator*{\argmin}{\arg\min}
\DeclareMathOperator*{\esssup}{\text{ess}\sup}

\DeclareMathOperator{\prob}{\mathbb{P}}
\DeclareMathOperator{\Cov}{\mathrm{Cov}}
\DeclareMathOperator{\TTM}{\textrm{TTM}}
\DeclareMathOperator{\diag}{\mathrm{diag}}
\DeclareMathOperator{\clip}{\mathrm{clip}}
\DeclareMathOperator{\cond}{\mathrm{cond}}
\DeclareMathOperator{\True}{\mathrm{True}}
\DeclareMathOperator{\sgn}{\mathrm{sgn}}
\DeclareMathOperator{\df}{\mathrm{d}\hspace{-0.07cm}}

\DeclareMathOperator{\comp}{\texttt{c}} %% for set complement

\DeclareMathOperator{\cA}{\mathcal{A}}
\DeclareMathOperator{\cB}{\mathcal{B}}
\DeclareMathOperator{\cC}{\mathcal{C}}
\DeclareMathOperator{\cD}{\mathcal{D}}
\DeclareMathOperator{\cE}{\mathcal{E}}
\DeclareMathOperator{\cF}{\mathcal{F}}
\DeclareMathOperator{\cG}{\mathcal{G}}
\DeclareMathOperator{\cH}{\mathcal{H}}
\DeclareMathOperator{\cI}{\mathcal{I}}
\DeclareMathOperator{\cJ}{\mathcal{J}}
\DeclareMathOperator{\cK}{\mathcal{K}}
\DeclareMathOperator{\cL}{\mathcal{L}}
\DeclareMathOperator{\cM}{\mathcal{M}}
\DeclareMathOperator{\cN}{\mathcal{N}}
\DeclareMathOperator{\cO}{\mathcal{O}}
\DeclareMathOperator{\cP}{\mathcal{P}}
\DeclareMathOperator{\cQ}{\mathcal{Q}}
\DeclareMathOperator{\cR}{\mathcal{R}}
\DeclareMathOperator{\cS}{\mathcal{S}}
\DeclareMathOperator{\cT}{\mathcal{T}}
\DeclareMathOperator{\cU}{\mathcal{U}}
\DeclareMathOperator{\cV}{\mathcal{V}}
\DeclareMathOperator{\cW}{\mathcal{W}}
\DeclareMathOperator{\cX}{\mathcal{X}}
\DeclareMathOperator{\cY}{\mathcal{Y}}
\DeclareMathOperator{\cZ}{\mathcal{Z}}

\DeclareMathOperator{\rA}{\mathrm{A}}
\DeclareMathOperator{\rB}{\mathrm{B}}
\DeclareMathOperator{\rC}{\mathrm{C}}
\DeclareMathOperator{\rD}{\mathrm{D}}
\DeclareMathOperator{\rE}{\mathrm{E}}
\DeclareMathOperator{\rF}{\mathrm{F}}
\DeclareMathOperator{\rG}{\mathrm{G}}
\DeclareMathOperator{\rH}{\mathrm{H}}
\DeclareMathOperator{\rI}{\mathrm{I}}
\DeclareMathOperator{\rJ}{\mathrm{J}}
\DeclareMathOperator{\rK}{\mathrm{K}}
\DeclareMathOperator{\rL}{\mathrm{L}}
\DeclareMathOperator{\rM}{\mathrm{M}}
\DeclareMathOperator{\rN}{\mathrm{N}}
\DeclareMathOperator{\rO}{\mathrm{O}}
\DeclareMathOperator{\rP}{\mathrm{P}}
\DeclareMathOperator{\rQ}{\mathrm{Q}}
\DeclareMathOperator{\rR}{\mathrm{R}}
\DeclareMathOperator{\rS}{\mathrm{S}}
\DeclareMathOperator{\rT}{\mathrm{T}}
\DeclareMathOperator{\rU}{\mathrm{U}}
\DeclareMathOperator{\rV}{\mathrm{V}}
\DeclareMathOperator{\rW}{\mathrm{W}}
\DeclareMathOperator{\rX}{\mathrm{X}}
\DeclareMathOperator{\rY}{\mathrm{Y}}
\DeclareMathOperator{\rZ}{\mathrm{Z}}

%% The operators \EE and \VV are starred because they denote 
%% the expectation and the variance operators respectively
%% The starred variant puts subscript below the letter (recall \lim)

\DeclareMathOperator{\AAA}{\mathbb{A}} %% \AA already defined
\DeclareMathOperator{\BB}{\mathbb{B}}
\DeclareMathOperator{\CC}{\mathbb{C}}
\DeclareMathOperator{\DD}{\mathbb{D}}
\DeclareMathOperator*{\EE}{\mathbb{E}} %% for 'proper' subscript
\DeclareMathOperator{\FF}{\mathbb{F}}
\DeclareMathOperator{\GG}{\mathbb{G}}
\DeclareMathOperator{\HH}{\mathbb{H}}
\DeclareMathOperator{\II}{\mathbb{I}}
\DeclareMathOperator{\JJ}{\mathbb{J}}
\DeclareMathOperator{\KK}{\mathbb{K}}
\DeclareMathOperator{\LL}{\mathbb{L}}
\DeclareMathOperator{\MM}{\mathbb{M}}
\DeclareMathOperator{\NN}{\mathbb{N}}
\DeclareMathOperator{\OO}{\mathbb{O}}
\DeclareMathOperator{\PP}{\mathbb{P}}
\DeclareMathOperator{\QQ}{\mathbb{Q}}
\DeclareMathOperator{\RR}{\mathbb{R}}
\DeclareMathOperator{\SSS}{\mathbb{S}} %% \SS already defined
\DeclareMathOperator{\TT}{\mathbb{T}}
\DeclareMathOperator{\UU}{\mathbb{U}}
\DeclareMathOperator*{\VV}{\mathbb{V}} %% for 'proper' subscript
\DeclareMathOperator{\WW}{\mathbb{W}}
\DeclareMathOperator{\XX}{\mathbb{X}}
\DeclareMathOperator{\YY}{\mathbb{Y}}
\DeclareMathOperator{\ZZ}{\mathbb{Z}}

\DeclareMathOperator{\avec}{\bf a}
\DeclareMathOperator{\bvec}{\bf b}
\DeclareMathOperator{\cvec}{\bf c}
\DeclareMathOperator{\dvec}{\bf d}
\DeclareMathOperator{\evec}{\bf e}
\DeclareMathOperator{\fvec}{\bf f}
\DeclareMathOperator{\gvec}{\bf g}
\DeclareMathOperator{\hvec}{\bf h}
\DeclareMathOperator{\ivec}{\bf i}
\DeclareMathOperator{\jvec}{\bf j}
\DeclareMathOperator{\kvec}{\bf k}
\DeclareMathOperator{\lvec}{\bf l}
\DeclareMathOperator{\mvec}{\bf m}
\DeclareMathOperator{\nvec}{\bf n}
\DeclareMathOperator{\ovec}{\bf o}
\DeclareMathOperator{\pvec}{\bf p}
\DeclareMathOperator{\qvec}{\bf q}
\DeclareMathOperator{\rvec}{\bf r}
\DeclareMathOperator{\svec}{\bf s}
\DeclareMathOperator{\tvec}{\bf t}
\DeclareMathOperator{\uvec}{\bf u}
\DeclareMathOperator{\vvec}{\bf v}
\DeclareMathOperator{\wvec}{\bf w}
\DeclareMathOperator{\xvec}{\bf x}
\DeclareMathOperator{\yvec}{\bf y}
\DeclareMathOperator{\zvec}{\bf z}

\DeclareMathOperator{\Avec}{\bf A}
\DeclareMathOperator{\Bvec}{\bf B}
\DeclareMathOperator{\Cvec}{\bf C}
\DeclareMathOperator{\Dvec}{\bf D}
\DeclareMathOperator{\Evec}{\bf E}
\DeclareMathOperator{\Fvec}{\bf F}
\DeclareMathOperator{\Gvec}{\bf G}
\DeclareMathOperator{\Hvec}{\bf H}
\DeclareMathOperator{\Ivec}{\bf I}
\DeclareMathOperator{\Jvec}{\bf J}
\DeclareMathOperator{\Kvec}{\bf K}
\DeclareMathOperator{\Lvec}{\bf L}
\DeclareMathOperator{\Mvec}{\bf M}
\DeclareMathOperator{\Nvec}{\bf N}
\DeclareMathOperator{\Ovec}{\bf O}
\DeclareMathOperator{\Pvec}{\bf P}
\DeclareMathOperator{\Qvec}{\bf Q}
\DeclareMathOperator{\Rvec}{\bf R}
\DeclareMathOperator{\Svec}{\bf S}
\DeclareMathOperator{\Tvec}{\bf T}
\DeclareMathOperator{\Uvec}{\bf U}
\DeclareMathOperator{\Vvec}{\bf V}
\DeclareMathOperator{\Wvec}{\bf W}
\DeclareMathOperator{\Xvec}{\bf X}
\DeclareMathOperator{\Yvec}{\bf Y}
\DeclareMathOperator{\Zvec}{\bf Z}

\DeclareMathOperator{\onevec}{\boldsymbol{1}}
\DeclareMathOperator{\zerovec}{\boldsymbol{0}}
\DeclareMathOperator{\thetavec}{\boldsymbol{\theta}}
\DeclareMathOperator{\muvec}{\boldsymbol{\mu}}
\DeclareMathOperator{\etavec}{\boldsymbol{\eta}}
\DeclareMathOperator{\sigmavec}{\boldsymbol{\sigma}}
\DeclareMathOperator{\pivec}{\boldsymbol{\pi}}
\DeclareMathOperator{\varrhovec}{\boldsymbol{\varrho}}
\DeclareMathOperator{\Lambdavec}{\boldsymbol{\Lambda}}
\DeclareMathOperator{\omegavec}{\boldsymbol{\omega}}

\DeclareMathOperator{\maj}{\textrm{majority}}
\DeclareMathOperator{\err}{\textrm{err}}
\DeclareMathOperator{\softmax}{\textrm{softmax}}
\DeclareMathOperator{\Log}{\textrm{Log}}
\DeclareMathOperator{\Ln}{\textrm{Ln}}
\DeclareMathOperator{\iid}{\textrm{iid}}
\DeclareMathOperator{\ER}{\textrm{ER}}
\DeclareMathOperator{\opt}{\textrm{opt}}

\DeclareMathOperator{\bigpipe}{\;\big\pipe\;}
\DeclareMathOperator{\Bigpipe}{\;\Big\pipe\;}
\DeclareMathOperator{\biggpipe}{\;\bigg\pipe\;}
\DeclareMathOperator{\Biggpipe}{\;\Bigg\pipe\;}

\DeclarePairedDelimiter{\set}{\{}{\}}
\DeclarePairedDelimiter{\size}{\lvert}{\rvert}
\DeclarePairedDelimiter{\innerprod}{\langle}{\rangle}
\DeclarePairedDelimiter{\ceil}{\lceil}{\rceil}
\DeclarePairedDelimiter{\floor}{\lfloor}{\rfloor}
\DeclarePairedDelimiter{\group}{(}{)}
\DeclarePairedDelimiter{\norm}{\|}{\|}

\NewDocumentCommand{\shivam}{m}{\textcolor{red}{(SG: #1)}}
